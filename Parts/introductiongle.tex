\newpage
\chapter*{Introductions génerale}
La révolution numérique et technologique actuelle procure des changements importants dans des différents secteurs particulièrement le secteur industriel qui connaît une croissance et un développement exponentiel à tous les niveaux ce processus de changement technologique a débuté avec la révolution industrielle qui a connu un basculement très fort d’une société artisanal vers une industrie commerciale et industrielle voir l’émergence d’industrie virtuelle en adoptant de nouvelle techniques. Les tendances actuelles exigent aux entreprises manufacturières qu’elles soient flexibles, compétentes et réactives et qu’elles s’adaptent rapidement face aux perturbations externes telles que les fluctuations du marché et aux perturbations internes telles que l’indisponibilité de ressources. 

Ce concept de progression permet de passer d’une politique tout à fait alentour de produit à une économie client (la production ne concentre plus sur le produit, mais plutôt sur les exigences du client.  C'est pour cela les entreprises tente à faire des changements au niveau de leur chaîne logistique et de leur système productif par l’utilisation de nouvelles stratégies, actualisation de nouveau procédé de fabrication en appuyant sur l’écoute du client afin de rester compétitive dans un marché très concurrentiel.). La bonne gestion de la chaîne logistique permet aux entreprises industrielles et commerciales, de se positionner dans un marché très concurrentiel et répondre de manière efficiente aux différentes fluctuations. Ces exigences permettront aux entreprises de confronter plusieurs défis, tels que, la réduction des délais de fabrication et de livraison, diminution des prix, et cela sans pour autant affecter la qualité des produits et la qualité de service. Dans le but de répondre aux exigences actuelles, il est impératif de veiller sur l’amélioration continue des performances de l’entreprise qui s’élabore par une bonne gestion de toutes les opérations au sein des entreprises. En effet à l’heure actuelle digitale, sont au cours tous les phénomènes, la vision de l’amélioration continue est déjà bien engagée, et les industriels deviennent plus concurrentiels, plus productifs et misent sur les opportunités et les performances a adapté au sein de leur entreprise pour une meilleure réactivité et un meilleur rendement.

Cette dynamique sensible du marché, rehausse constamment les normes des industriels qui souhaitent accroître leur part de marché et satisfaire leurs clients. A cet égard, les erreurs et défauts de production de toutes causes confondues sont peu tolérés voir même intolérés chez certains producteurs en donnant une grande importance au contrôle de la qualité tout au long du processus. 

Pour répondre à ce contexte, les différents acteurs se tournent vers le développement des solutions innovantes facilitant les activités de production et le repérage des différents dysfonctionnements et des défauts de qualité. Le concept de l’industrie 4.0 se trouve au cœur de ces changements par la mobilisation des nouvelles technologies telles que les IOT (Internet Of Things), l’intelligence artificielle (IA) et le deep learning etc. 

L'intelligence artificielle a introduit plusieurs technologies dans l'industrie, évidemment dans le concept de qualité et de détection des défauts. L'IA à travers ses composants, notamment l'apprentissage profond, a donné naissance à des applications de vision artificielle spécialisées dans la détection de ces défauts de production.

La tendance dans ces applications est d'atteindre une précision de niveau humain ou mieux dans l'inspection de la qualité. Cela implique un minimum d'erreurs dans la production. Et donc une bonne réputation de l'entreprise sur le marché.
L'objectif de ce travail est de proposer une application de détection de non-conformité par apprentissage profond appliquée à un système de production alimentaire de nouilles référencé à la société JUMBO.

Le travail présenté dans ce mémoire se décompose en deux partie, la première abordera une recherche bibliographique sur la qualité et l’inspection de la qualité dans le contexte de l’industrie 4.0, suivant trois chapitres comme suit :
\begin{itemize}
    \item Un premier chapitre est  dédié  a présenté les notions de  contrôle de la qualité dans l’industrie 4.0  afin de  positionner  notre abordée dans le contexte de l’industrie 4.0.
    \item Un deuxième chapitre porte sur l’intelligence artificielle et ses composantes pour expliquer la science derrière les applications de l’industrie 4.0.
    \item Un troisième chapitre sur les systèmes de vision artificielle et le traitement d'images qui décrira tous les outils nécessaires pour construire une application de détection des défauts par vision artificielle.
\end{itemize}



La deuxième partie de ce travail concerne nos apports et contributions dans ce projet est décomposée également en deux chapitres comme suit : le quatrième chapitre sur le cadre méthodologique, et le cinquième  chapitre sur l’étude comparative et le choix du modèle.

Ce projet final est le résultat d'une tentative d'utiliser autant que possible les connaissances et les capacités que nous avons acquises au cours de notre programme éducatif. L'entreprise RMBtech Smart Automation, qui nous a accompagnés dans ce projet, nous a offert le cadre et l'environnement pratiques dans lesquels nous avons travaillé.
